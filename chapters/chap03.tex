% !TeX root = ../ahuthesis-example.tex

\chapter{数学符号和公式}

\section{数学符号}

中文论文的数学符号默认遵循 GB/T 3102.11--1993《物理科学和技术中使用的数学符号》
\footnote{原 GB 3102.11—1993, 自 2017 年 3 月 23 日起, 该标准转为推荐性标准.}.
具体地来说主要有以下差异:
\begin{enumerate}
  \item 实部 $\Re$ 和虚部 $\Im$ 的字体使用罗马体.
\end{enumerate}

另外国标还有一些与 AMS 不同的符号使用习惯, 需要用户在写作时进行处理:
\begin{enumerate}
  \item 数学常数和特殊函数名用正体, 如
    \begin{equation*}
      \uppi = 3.14\dots; \quad
      \symup{i}^2 = -1; \quad
      \symup{e} = \lim_{n \to \infty} \left( 1 + \frac{1}{n} \right)^n.
    \end{equation*}
  \item 微分号使用正体, 比如 $\dif y / \dif x$.
  \item 自然对数用 $\ln x$ 不用 $\log x$.
\end{enumerate}

关于量和单位推荐使用 \pkg{siunitx} 宏包, 可以方便地处理希腊字母以及数字与单位之间的空白, 比如:
\SI{6.4e6}{m}, \SI{9}{\micro\meter}, \si{kg.m.s^{-1}}, \SIrange{10}{20}{\degreeCelsius}.


\section{数学公式}

数学公式可以使用 \env{equation} 和 \env{equation*} 环境。
注意数学公式的引用应前后带括号, 通常使用 \cs{eqref} 命令, 比如式 \eqref{eq:example}.
\begin{equation}
  \frac{1}{2 \uppi \symup{i}} \int_\gamma f = \sum_{k=1}^m n(\gamma; a_k) \mathscr{R}(f; a_k).
  \label{eq:example}
\end{equation}

多行公式尽可能在“=”处对齐, 推荐使用 \env{align} 环境.
\begin{align}
  a & = b + c + d + e \\
    & = f + g
\end{align}


\section{数学定理}

定理环境的格式可以使用 \pkg{amsthm} 或者 \pkg{ntheorem} 宏包配置.
用户在导言区载入这两者之一后, 模板会自动配置 \env{theorem}, \env{proof} 等环境.

\begin{theorem}[Lindeberg--Lévy 中心极限定理]
  设随机变量 $X_1, X_2, \dots, X_n$ 独立同分布, 且具有期望 $\mu$ 和有限的方差 $\sigma^2 \ne 0$,
  记 $\bar{X}_n = \frac{1}{n} \sum_{i+1}^n X_i$,则
  \begin{equation}
    \lim_{n \to \infty} P \left(\frac{\sqrt{n} \left( \bar{X}_n - \mu \right)}{\sigma} \le z \right) = \Phi(z),
  \end{equation}
  其中 $\Phi(z)$ 是标准正态分布的分布函数。
\end{theorem}
\begin{proof}
  Trivial.
\end{proof}

同时模板还提供了 \env{assumption}、\env{definition}、\env{proposition}、
\env{lemma}、\env{theorem}、\env{axiom}、\env{corollary}、\env{exercise}、
\env{example}、\env{remar}、\env{problem}、\env{conjecture} 这些相关的环境.
