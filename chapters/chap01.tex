% !TeX root = ../ahuthesis-example.tex

\chapter{论文的基本要求及内容}

学位论文是标明作者从事科学研究取得的创造性成果和创新见解, 并以此为内容撰写的、作
为申请学位时评审用的学术论文. 硕士学位论文应表明作者在本门学科上掌握了坚实的基础
理论和系统的专门知识, 对所研究的课题有新的见解, 并具有从事科学研究工作或独立担负
专门技术工作能力. 博士学位论文应表明作者在本门学科上掌握了坚实宽广的基础理论和系
统深入的专门知识, 在科学和专门技术上做出了创造性的成果, 并具有独立从事科学研究工
作的能力. 为提高研究生学位论文的质量, 做到学位论文在内容和格式上的统一和规范, 根
据《中华人民共和国国家标准科学技术报告、学位论文和学术论文的编写格式》(GB/T 7713.1--2006)
的规定, 特制定《安徽大学研究生撰写学位论文的规定》.


\section{论文的基本要求}

论文应立论正确、推理严谨、说明透彻、数据可靠. 论文应结构合理、层次分明、叙述准确、
文字简练、文图规范. 对于涉及作者创新性工作和研究特点的内容应重点论述, 做到数据或
实例丰富、分析全面深入. 文中引用的文献资料必须注明来源, 使用的计量单位、绘图规范
应符合国家标准. 论文的学术水平应满足《安徽大学申请博士、硕士学位学术成果基本要求》的要求.


\section{论文内容}

包括: 选题的背景、依据及意义; 文献及相关研究综述、研究及设计方案、试验方法、装置和试验结果;
理论的证明、分析和结论;重要的计算、数据、图表、曲线及相关分析; 必要的附录、相关的参考文献目录等.

对于合作完成的项目, 论文的内容应侧重本人的研究工作. 论文中有关与指导教师或他人共同
研究、试验的部分以及引用他人研究成果的部分都要明确说明.


\section{论文的书写规范与打印要求}

\subsection{论文的文字}

除英文封面、英文摘要外, 研究生学位论文的其余部分都应该用中文撰写, 以下两种情况除外:

(1) 留学生学位论文的目录、正文和致谢等可用英文撰写; 但封面、题名页、独创性声明和使用授权书应用中文撰写, 摘要应中英文对照撰写.

(2) 外语专业的学位论文的目录、正文和致谢等应用所学专业相应的语言撰写; 但封面、题名页、独创性声明和使用授权书应用中文撰写, 
摘要应使用中文和所学专业相应的语言对照撰写. 

\subsection{论文的书写}

学位论文一律由本人在计算机上输入、编排并打印在标准 A4 纸上, 封面(中、英文)、题名页、独创性声明和使用授权书采用单面印刷,
从中文摘要开始采用双面印刷(总页数少于 100 页的学位论文, 为了制作书脊的需要, 可以采用单面印刷, 单面印刷可以允许用 
$210\times 297$mm, 70g 的纸张). 应该便于阅读、复制和拍摄缩微制品.


\section{论文的主要结构及装订顺序}

学位论文一般应由 $12$ 个部分组成, 装订顺序依次为:

(1) 封面(中、英文)	

(2) 题名页	

(3) 独创性声明和使用授权书	

(4) 中文摘要	

(5) 英文摘要	

(6) 目录

(7) 图表清单及主要符号表(根据具体情况可省略)	

(8) 主体部分

(9) 参考文献

(10) 附录

(11) 攻读博士学位期间参与的科研项目和取得的研究成果/攻读硕士学位期间参与的科研项目和取得的学术成果

(12) 致谢