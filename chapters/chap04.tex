% !TeX root = ../ahuthesis-example.tex

\chapter{引用文献的标注}

模板支持 BibTeX 方式处理参考文献. 下文主要介绍 BibTeX 配合 \pkg{natbib} 宏包的主要使用方法.


\section{顺序编码制}

在顺序编码制下, 默认的 \cs{cite} 命令同 \cs{citep} 一样, 序号置于方括号中, 
引文页码会放在括号外. 统一处引用的连续序号会自动用短横线连接.

{\ahusetup{
  cite-style = super,
}
\noindent
\begin{tabular}{l@{\quad$\Rightarrow$\quad}l}
  \verb|\cite{Lovasz2012}|               & \cite{Lovasz2012}               \\
  \verb|\citet{Lovasz2012}|              & \citet{Lovasz2012}              \\
  \verb|\citep{Lovasz2012}|              & \citep{Lovasz2012}              \\
  \verb|\cite{Lovasz2012,Terpai2011}|    & \cite{Lovasz2012,Terpai2011} \\
\end{tabular}}

也可以取消上标格式, 将数字序号作为文字的一部分. 建议全文统一使用相同的格式.

{\ahusetup{
  cite-style = inline,
}
\noindent
\begin{tabular}{l@{\quad$\Rightarrow$\quad}l}
  \verb|\cite{Lovasz2012}|               & \cite{Lovasz2012}               \\
  \verb|\citet{Lovasz2012}|              & \citet{Lovasz2012}              \\
  \verb|\citep{Lovasz2012}|              & \citep{Lovasz2012}              \\
  \verb|\cite{Lovasz2012,Terpai2011}|    & \cite{Lovasz2012,Terpai2011} \\
\end{tabular}}


\section{著者-出版年制}

著者-出版年制下的 \cs{cite} 跟 \cs{citet} 一样.

{\ahusetup{
  cite-style = author-year,
}
\noindent
\begin{tabular}{@{}l@{$\Rightarrow$}l@{}}
  \verb|\cite{Lovasz2012}|                & ~~\cite{Lovasz2012}                \\
  \verb|\citet{Lovasz2012}|               & ~~\citet{Lovasz2012}               \\
  \verb|\citep{Lovasz2012}|               & ~~\citep{Lovasz2012}               \\
  \verb|\citep{Lovasz2012,Terpai2011}|    & ~~\citep{Lovasz2012,Terpai2011} \\
\end{tabular}}


\ahusetup{
  cite-style = super,
}
注意, 引文参考文献的每条都要在正文中标注
\cite{Balogh-Clemen2021,Cioaba2021,dupont1974bone,jianduju1994,%
jiangxizhou1980,Lovasz2012,Nosal1970,scitor2000project,Terpai2011,%
tushuguan1957tushuguanxue,TuXuWang1987}