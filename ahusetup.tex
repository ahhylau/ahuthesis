% !TeX root = ./ahuthesis-example.tex

% 论文基本信息配置

\ahusetup{
  %******************************
  % 注意:
  %   1. 配置里面不要出现空行
  %   2. 不需要的配置信息可以删除
  %   3. 建议先阅读文档中所有关于选项的说明
  %******************************
  %
  % 输出格式
  %   选择打印版 (print) 或用于提交的电子版 (electronic), 前者会插入空白页以便直接双面打印
  %
  output = print,
  %
  % 标题 (可使用“\\”命令手动控制换行)
  %
  title  = {安徽大学学位论文 \LaTeX{} 模板使用示例文档 v\version},
  title* = {An Introduction to \LaTeX{} Thesis Template of Anhui
            University v\version},
  %
  %%%%%%%%%%%%%%%%%%%%%%%%%%% 博士学位论文信息填写 %%%%%%%%%%%%%%%%%%%%%%%%%%%
  clc              = {O157.5},                % 中图分类号
  degree-category  = {理学博士},
  degree-category* = {Doctor of Philosophy},
  %
  department = {数学科学学院},                 % 培养单位(填写所属院系的全名)
  discipline  = {数学},                       % 一级学科(中文)
  discipline* = {Mathematics},               % 一级学科(英文)
  sub-discipline = {基础数学},                % 二级学科
  author  = {作者姓名},                       % 作者姓名(中文)
  author* = {Name},                          % 作者姓名(英文)
  student-id = {A20614045},                  % 学号
  supervisor  = {xxx},                       % 指导教师(中文姓名和职称之间以英文逗号“,”分开, 下同)
  supervisor* = {xxx},
  professional-rank = {教授},                % 指导教师的职称
  % 提名页信息填写
  start-date = {2015-09-01},                % 学习开始日期
  end-date   = {2019-02-01},                % 学习结束日期
  date       = {2019-06-01},                % 论文提交日期
  defense-date = {2019-05-01},              % 论文答辩日期
  % 是否在中文封面后的空白页生成书脊(默认 false)
  include-spine = true,
  spine-date = {2025},
  %
  % 
  %%%%%%%%%%%%%%%%%%%%%%%%%%% 硕士信息填写 %%%%%%%%%%%%%%%%%%%%%%%%%%%
  % clc              = {O157.5},                % 中图分类号
  % degree-category  = {理学硕士},
  % degree-category* = {Doctor of Philosophy},
  % %
  % department = {数学科学学院},                 % 培养单位(填写所属院系的全名)
  % discipline  = {数学},                       % 一级学科(中文)
  % discipline* = {Mathematics},               % 一级学科(英文)
  % sub-discipline = {基础数学},                % 二级学科
  % author  = {作者姓名},                       % 作者姓名(中文)
  % author* = {Name},                      % 作者姓名(英文)
  % student-id = {A20614045},                  % 学号
  % supervisor  = {xxx},                       % 指导教师(中文姓名和职称之间以英文逗号“,”分开, 下同)
  % supervisor* = {xxx},
  % professional-rank = {教授},                % 指导教师的职称
  % % 提名页信息填写
  % start-date = {2015-09-01},                % 学习开始日期
  % end-date   = {2019-02-01},                % 学习结束日期
  % date       = {2019-06-01},                % 论文提交日期
  % defense-date = {2019-05-01},              % 论文答辩日期
  % % 是否在中文封面后的空白页生成书脊(默认 false)
  % include-spine = true,
  % spine-date = {2025},
  %
  %
  %%%%%%%%%%%%%%%%%%%%%%%%%%% 博士后出站报告信息填写 %%%%%%%%%%%%%%%%%%%%%%%%%%%
  %
  % clc            = {O157.5},           % 中图分类号
  % udc            = {UDC},
  % id             = {编号},
  % discipline     = {一级学科},          % 流动站(一级学科)名称
  % sub-discipline = {二级学科},          % 专业(二级学科)名称
  % start-date     = {2011-07-01},       % 研究工作起始时间
}
  %
  % 
%%%%%%%%%%%%%%%%%%%%%%%%%%% 载入所需的宏包 %%%%%%%%%%%%%%%%%%%%%%%%%%%
% 定理类环境宏包
\usepackage{amsthm}
% 也可以使用 ntheorem
% \usepackage[amsmath,thmmarks,hyperref]{ntheorem}

\ahusetup{
  % 数学字体
  math-font  = newcm,  % xits | newcm
  cite-style = super
}

% 表格加脚注
\usepackage{threeparttable}

% 跨页表格
\usepackage{longtable}

% 算法
\usepackage{algorithm}
\usepackage{algorithmic}

% 量和单位
\usepackage{siunitx}

% 参考文献使用 BibTeX + natbib 宏包
% 顺序编码制
\usepackage[sort]{natbib}
\bibliographystyle{ahuthesis-numeric}

% 著者-出版年制
% \usepackage{natbib}
% \bibliographystyle{ahuthesis-author-year}

% 定义所有的图片文件在 figures 子目录下
\graphicspath{{figures/}}

% 数学命令
\makeatletter
\newcommand\dif{%  % 微分符号
  \mathop{}\!\mathrm{d}%
}
\makeatother

% hyperref 宏包在最后调用
\usepackage{hyperref}